%!TEX root = info-main.tex
%%% Section IV
\section{Detection Architecture in a Multi-AP Network}

In this section, we first identify the key features of selfish
carrier sensing behavior. We then propose an online detection
architecture, called {\tt CUBIA}, that diagnoses the network
condition in a multi-AP network, consisting of (i) a \emph{AP-level
diagnosis} stage and (ii) a \emph{system-level reaction} stage.
%
In what follows, we will detail the design of this architecture.


\subsection{Key Feature of Selfish Carrier Sense Behavior}

Selfish carrier sensing on a tethered host node may result in
excessive frame loss of nearby legitimate APs, and thus, successful
detection of selfish misbehavior depend largely on a AP's ability
of identifying the root cause of its frame losses.
%
In 802.11 WLANs, there are four main causes of frame losses:
(i) PHY-layer link quality degradation, (ii) MAC-layer collision,
(iii) hidden nodes, and (iv) selfish nodes (e.g., manipulation
of the CCA thresholds).

Fortunately, selfish tethering exhibits unique features that
can help distinguishing the selfish carrier sensing of a
tethering host from other causes of frame losses, especially
from the hidden node problem.
%
We made the following two observations:

\myitemizebegin
%
\item[{\bf O1)}]
While the victim nodes of the selfish tethering experience
symptoms similar to those of the \emph{hidden node problem}
(i.e., transmissions of the legitimate node are continuously
corrupted by the selfish node), the victim nodes' communications
cannot be fully protected by the Request-to-Send/Clear-to-Send
(RTS/CTS) exchange mechanism, i.e., a virtual carrier sensing,
because the selfish node may not recognize (due to its short
communication range) or respect the RTS/CTS frames.
%
%Unlike the hidden node problem, however, remedies are different.
%That is, the hidden node problem can be mitigated by the
%Request-to-Send/Clear-to-Send (RTS/CTS) exchange, i.e., the virtual
%carrier sensing mechanism, but the selfish carrier sensing problem cannot,
%because the RTS/CTS frames are not recognized/respected by the selfish
%node due to its short communication range.
%
\item[{\bf O2)}]
When selfish tethering exists, the victim nodes may suffer from
increased number of frame losses, hence throughput degradation,
because the selfish tethering initiates a transmission even when
the victim nodes have already been transmitting packets.
%
In a multi-AP network, majority of nearby APs will likely to
experience the similar severe packet errors simultaneously when
the selfish tethering is present. In contrast, the impact of the
hidden node problem is limited to the nodes in the hidden area.
%
\myitemizeend


\subsection{CUBIA: A Proposed Selfish Diagnosis Algorithm}

Based on the above observations, we propose a simple, yet efficient
online detection algorithm at each AP, which can accurately distinguish
the selfish behavior with CCA manipulation from other types of network
problems, such as severe network congestion (i.e., collisions) and
the hidden node problem.

\subsubsection{Diagnosis Stage}
%
The first diagnosis stage is to infer the asymmetric carrier sensing
condition at each AP in the network. Based on above observations,
we propose a simple interference inference algorithm, called {\tt CUBIA}
({\em Two-Phase CUsum-Based Interference inference with Adaptive RTS/CTS}).
%
%By exploiting the above observation on the unique feature of selfish
%misbehavior using CCA manipulation, we propose a simple interference
%inference algorithm, so called CUBIA ({\em Two-Phase CUsum-Based
%Interference inference with Adaptive RTS/CTS}).
%
{\tt CUBIA} runs at each AP and it diagnoses the network condition by passively
monitoring the ongoing traffic with its client nodes.
%
Specifically, if {\tt CUBIA} detects severe and persistent frame transmission
failures, it employs the RTS/CTS exchange before each data transmission
to rule out the possibility of the hidden node problem.
%
%differentiate the hidden node problem from frame transmission failures
%by the selfish carrier sensing.
%
For this, {\tt CUBIA} employs the CUSUM (CUmulative SUM) algorithm~\cite{Gustafsson:2000}
to quantify the duration of frame losses. CUSUM algorithm suits our needs
because it is simple and light-weight, and it has been widely used for the
detection of state changes. We will elaborate the detailed procedure of
the detection mechanism using CUSUM algorithm.

%As mentioned earlier, unlike the hidden node problem,
%a data transmission following a successful RTS/CTS exchange
%is not protected in the presence of a selfish node.
%The heart of CUBIA algorithm is that,
%to quantify how long the corruption lasts, it adopts the CUSUM
%(CUmulative SUM) algorithm~\cite{Gustafsson:2000}, which has been
%widely used for the detection of state changes.
%
% Jaehyuk: Due to the lack of 80211n node in simulation tools,
%                   I changed the algorithm that is available in ns-2
%
%In particular, the AP exploits the frame aggregation and
%block ACK mechanism in the 802.11 to diagnose the cause of
%frame losses~\cite{ieee:80211e, ieee:80211n}.
%
%Frame aggregation is a feature of the IEEE 802.11e and 802.11n standards,
%designed to increase throughput by sending two or more data frames
%in a single transmission~\cite{ieee:80211e, ieee:80211n}.
%%
%802.11e EDCA provides contention-free access to the channel for a period
%called {\em Transmit Opportunity} (TXOP), allowing multiple frames
%to be transmitted in a single TXOP.
%%
%The 802.11n standard allows MAC Protocol Data Unit (MPDU) aggregation,
%called A-MPDU, whereby multiple subframes are packed into
%a single aggregated frame.
%%
%Block acknowledgments (block ACKs) allow an entire TXOP and A-MPDU
%to be acknowledged in a single frame, reducing protocol overhead.

%We made the following observation to detect the asymmetric carrier
%sensing condition: in case of asymmetric carrier sensing condition
%like the hidden node problem, the receiver of the ongoing transmission
%of an aggregated frame (e.g., A-MPDU) can suffer from \emph{partial}
%and \emph{bursty} frame losses.
%
%
%%% alex: why this is the case?
%In particular, bit errors are likely to be located after the front
%section of the aggregated transmission, while the front section of the
%transmission has a much lower probability of incurring bit errors
%compared to the rest of the packet.
%
%
%%
%%  [FIGURE]
%%Fig.~X shows the
%%[Experiment results should be here]
%
%Then, the receiver reports the status (i.e., success/failure) of each
%subframe of the A-MPDU frame to the transmitter via a block ACK frame.
%Consequently, the transmitter node can know the result of the
%transmission and calculate the error probability of subframes.

{\tt CUBIA} monitors the frame error rate (FER) for every $m$ transmissions
to detect any abrupt changes in network condition, which might indicate
the possibility of selfish tethering nodes.
%
Let $p_{k}^{i}$ denote the frame error rate for the $k$-th measurement
period at $AP_i$, given by $p_{k}^{i} = e_{k}/m$, where $e_{k}$ denotes
the number of transmission failures. Then, we calculate the average
error probability, $E[p^{i}]$, by a moving average to reflect the
network dynamics as:
%
\begin{equation} \label{Eq:move_avg}
E[p^{i}] = \lambda \cdot E[p^{i}]  + (1 - \lambda) \cdot p_{k}^{i}.
\end{equation}

We define the CUSUM change detection filter $c_k^i$ of $AP_i$ as:
\begin{eqnarray}
&c_k^i = \max(~0, ~c_{k-1}^i + p_{k}^{i} - v~),\label{Eq:CUSUMg} \\
&c_0^i = 0,\nonumber
\end{eqnarray}
where $v$ is a drift parameter, which is a filter design parameter.
$v$ is configured differently according to the value of $c_k^i$ as:
%
\begin{equation}\label{Eq:CUSUMv}
v =
\begin{cases}
E[p^{i}] + \mathcal{T}_{FER},  & \mbox{if } c_k^i < \theta_{AS} \\
\mathcal{T}_{FER}, & \mbox{if } \theta_{AS} \leq  c_k^i \leq \theta_{S},
\end{cases}
\end{equation}
%
where $\theta_{AS}$ and $\theta_{S}$ denote
the {\em first alarm threshold} for asymmetric carrier sensing, and
the {\em second alarm threshold} for inferring selfish carrier sensing,
respectively.

{\tt CUBIA} issues the first alarm to $AP_i$ when $c_k^i > \theta_{AS}$.
Then, {\tt CUBIA} turns on the RTS/CTS exchange mechanism, and
keeps track of the change detection with
$v = \mathcal{T}_{FER}$ in Eq.~\eqref{Eq:CUSUMg}.
%
Note that the first alarm can be caused by
a sudden severe MAC-layer congestion or a hidden node problem.
%
%
However, in such a case,
%i.e., if the first alarm stems from the MAC-layer congestion or
%the hidden terminal problem,
the magnitude of $c_k^i$ tends to decrease with RTC/CTS frames,
because the the collisions are filtered out \cite{CARA} and
the hidden terminal problem are mitigated with RTS/CTS
exchange, and the frame error rate decreases accordingly.

On the other hand, if the first alarm is caused by the selfish carrier
sensing problem, the transmission of a frame transmission following
successful RTS/CTS exchanges will be interfered with, and hence,
 even with RTS/CTS, the magnitude of $c_k^i$ would continue to
increase, even beyond $\theta_{S}$.
%
%
Recall that under the normal condition, PHY-layer link quality is stable
and has a certain upper bound $\mathcal{T}_{FER}$, namely \emph{target}
frame error rate, which is guaranteed by the underlying rate-adaptation
scheme that adjusts the modulation schemes to meet the target
FER.\footnote{For example, practical rate adaptations \cite{Jensen10}
adjust the modulation schemes to meet the target FER (frame error rate),
and thus guarantee the average FER performance to be maintained
around the target value.}

Consequently, $AP_i$ sends a second alarm to the central controller
(see Fig.~\ref{fig-architecture}) if $c_k^i > \theta_{S}$,
which may trigger the \emph{reaction}
to solve the selfish carrier sensing problem.

\subsubsection{Reaction Stage}
%
Although the focus of this paper is to propose an AP-level detection
mechanism, it is important to cope with the selfish misbehavior detected
whin the network at the system level.
% 
%Here, we present a simple decision policy with which the controller
%determines when to take follow-up actions to cope with the selfish
%misbehavior detected whin the network.
% Jaehyuk
Here, we discuss how the controller determines when to take 
follow-up actions to cope with the selfish misbehavior detected 
whin the network.

In typical managed Wi-Fi networks, the information obtained at APs
is integrated on the controller. Then, the controller utilizes
the information to improve the detection accuracy.
%
As explained in observation {\bf O2}, majority of nearby APs will likely 
to experience the similar severe packet errors simultaneously when
the selfish tethering is present. Consequently, the victim APs send
the second alarm to the controller at the same time.
By exploiting the spatial and temporal correlation in those alarms, 
the controller identifies the root of the problem more effectively and 
accurately. 
%
For example, if a certain condition is satisfied,
the controller can take various follow-up actions, which can
be (i) localizing the rogue interfering node~\cite{Joshi:Katti2013},
and (ii) remedying victim APs (e.g., interference-aware channel
reassignment), etc. However, the detailed follow-up actions are
beyond the scope of this paper.
