\section{Related Work}

Selfish and malicious misbehavior in CSMA networks
has been extensively studied 
under various scenarios in different communication layers.
\cite{Kysanur:Vaidya05, Radosavac-Wise05, Raya-DOMINO, Toledo:Wang07a}.
%However,
%to the best of our knowledge, this is the first work that
%addresses the selfish misbehavior exploiting
%the Wi-Fi tethering environment.

%MAC-layer
Majority of previous work has concentrated on detecting or punishing
MAC-layer misbehavior \cite{Kysanur:Vaidya05, Raya-DOMINO, Radosavac-Wise05, Rong-Infocom06, Toledo:Wang07a}.
%
The authors of \cite{Kysanur:Vaidya05} have addressed the MAC layer misbehavior
selecting always small backoff values rather than random selection
and showed that such selfish misbehavior
can seriously degrade the performance of the network. To detect and penalize
the selfish misbehavior, they modified the MAC protocol and proposed a
receiver-side backoff allocation strategy where let
the receiver assign and send backoff values to the sender in
CTS and ACK frames.
Then, the assigned receiver uses them to detect potential misbehavior.
However, it needs to modification on the IEEE
802.11 protocol and the selfish receiver may exist.
%
In \cite{Radosavac-Wise05},
several possible misbehavior in 802.11 MAC protocol including
backoff manipulation are presented.
A detection framework, named DOMINO, considers
every possible strategies jointly and increases the detection accuracy.
The detection scheme for backoff manipulation is based on comparing
average values of the backoff to given thresholds.
However, this average metric is a suboptimal detection technique easily
misled by extremely large values of samples.
Some previous works~\cite{Toledo:Wang07a, Radosavac-Wise05} have utilized
statistical testing techniques such as
the Wald��s sequential probability ratio test~(SPRT)~\cite{Radosavac-Wise05},
Kolmogorov-Smirnov (K-S) test~\cite{Toledo:Wang07a}.
These works were important sources of inspiration for our work.
However, their operation cost is much higher
than our detection mechanism since their frameworks are
based on the comparison of distributions and
need to construct the probability distributions from
the observations. Moreover, no operational method to mitigate misbehavior
is proposed.


%Physical-layer
Some previous work has focused on routing layer misbehavior\cite{BenSalem03, Marti03}.
In 

% Routing-layer
Various solutions to routing layer misbehavior
(e.g., \cite{BenSalem03, Marti03})
and game theory-based approaches~\cite{MacKenzie01,Akella02} in wireless
ad hoc networks have been proposed.

%Multi-channel


%CCA Threshold





In our previous work,


A common way for a selfish user to increase his chance of accessing
the channel is to modify the operation of the 802.11 protocol
or change the MAC parameters related with
channel-attempt parameters, such as $CW_{min}$,
$CW_{max}$, and IFS~(inter-frame space).
Kyasanur and Vaidya \cite{Kysanur:Vaidya05} studied the MAC-layer misbehavior
of selecting always small backoff values rather than
random selection, and showed that such selfish misbehavior can
seriously degrade the network performance.
Raya {\em et al.}~\cite{Raya-DOMINO} investigated
multiple misbehavior policies in 802.11 MAC protocol including
backoff manipulations, and presented a detection framework, called DOMINO,
which considers all possible strategies jointly and improves the detection accuracy.
Recently, several statistic-based frameworks for detecting misbehavior
also have been introduced \cite{Radosavac-Wise05, Rong-Infocom06, Toledo:Wang07a}.
The common approach used in most prior works\cite{Kysanur:Vaidya05, Radosavac-Wise05,
Raya-DOMINO, Toledo:Wang07a} is to measure the inter-arrival time of target stations
in terms of the number of backoff slots and verify whether the backoff time of the stations follows a legitimate pattern or not.


Protocol misbehavior in 802.11 networks has been studied
in various scenarios in different communication layers.
The authors of \cite{Kysanur:Vaidya05} have addressed the MAC layer misbehavior
selecting always small backoff values rather than random selection
and showed that such selfish misbehavior
can seriously degrade the performance of the network. To detect and penalize
the selfish misbehavior, they modified the MAC protocol and proposed a
receiver-side backoff allocation strategy where let
the receiver assign and send backoff values to the sender in
CTS and ACK frames.
Then, the assigned receiver uses them to detect potential misbehavior.
However, it needs to modification on the IEEE
802.11 protocol and the selfish receiver may exist.
%
In \cite{Radosavac-Wise05},
several possible misbehavior in 802.11 MAC protocol including
backoff manipulation are presented.
A detection framework, named DOMINO, considers
every possible strategies jointly and increases the detection accuracy.
The detection scheme for backoff manipulation is based on comparing
average values of the backoff to given thresholds.
However, this average metric is a suboptimal detection technique easily
misled by extremely large values of samples.
Some previous works~\cite{Toledo:Wang07a, Radosavac-Wise05} have utilized
statistical testing techniques such as
the Wald��s sequential probability ratio test~(SPRT)~\cite{Radosavac-Wise05},
Kolmogorov-Smirnov (K-S) test~\cite{Toledo:Wang07a}.
These works were important sources of inspiration for our work.
However, their operation cost is much higher
than our detection mechanism since their frameworks are
based on the comparison of distributions and
need to construct the probability distributions from
the observations. Moreover, no operational method to mitigate misbehavior
is proposed.

% Routing-layer
Various solutions to routing layer misbehavior
(e.g., \cite{BenSalem03, Marti03})
and game theory-based approaches~\cite{MacKenzie01,Akella02} in wireless
ad hoc networks have been proposed. However, as the problem we
consider in this paper focuses on the MAC layer.

%Multi-channel

%%
The problem of misbehavior using CCA threshold is relatively new
and unexplored in the literature.
Recently, the authors~\cite{yang:choi-LCN09} have identified the effect of an
802.11 node's selfish behavior by increasing CCA threshold.
They have shown that the selfish behavior can achieve
higher throughput than other rational nodes based on
proposed game-theoretical model.
In \cite{Pelechris:Krish-Infocom09}, the selfish carrier sense problem
in WLANs was addressed. The authors have performed experimentation
on a real testbed and shown that such selfish behaviors can
cause extremely unfair allocations of the wireless medium.
We develop a detection scheme, named Carrier
sensing Misbehavior Detectio (CMD),
by exploiting power control mechanism.


