%!TEX root = info-main.tex
\section{Introduction}

%Driven by the huge success of smartphones,
%Wi-Fi tethering has recently been recognized
%as an attractive means to provide Internet connectivity.
%The Wi-Fi tethering refers to sharing
%the Internet connection of mobile devices, such as smartphones and
%tablets, with another Wi-Fi-enabled devices through
%Wi-Fi~\cite{Choi:Shin11}.
%%
%One of the major benefits of tethering with a smartphone is
%that it enables users to go online almost anywhere, even on-the-move.
%
With the rapidly-increasing deployment of mobile devices and skyrocketing
demand for mobile data traffic~\cite{Cisco:13}, Wi-Fi has been the
most widely-available and energy-efficient means to provide the Internet
access to mobile users and devices~\cite{Balasubramanian:09}.
%
While most of the today's mobile devices are equipped with Wi-Fi
interfaces, supporting very high throughput performance even with
small form-factor smartphones, e.g., up to 433\,Mbps for
802.11ac,\footnote{The new 802.11ac standard supports up to several
Gbps of throughput depending on transmission configurations, e.g.,
the number of antennae and channel bandwidth.} the limited and short
coverage of Wi-Fi networks restrict their usage.
%
For example, a mobile user may not be able to find a Wi-Fi access
point (AP) while traveling by a car on a highway. Or a mobile
user in a coffee shop may find an AP with very limited bandwidth
because many people share the same AP.
%
To address these issues, Wi-Fi tethering has recently been gaining
popularity as a means to provide Internet connectivity by sharing the
Internet connection of mobile devices with another Wi-Fi-enabled devices
through Wi-Fi~\cite{Choi:Shin11}.

While Wi-Fi tethering allows mobile users to go online almost anywhere,
even on-the-move, the ease of setting up a tethered Wi-Fi hotspot
and the open source nature of mobile Operating Systems (OSes) can pose
serious performance problems to well-planned Wi-Fi networks, such as
enterprise and campus networks~\cite{Himanshu2010}.
%
Since the tethered Wi-Fi hotspot can potentially open up the network
with an arbitrary channel number without restriction from the network
administrator, it may interfere with nearby well-planned APs and cause
unpredictable performance degradation, e.g., availability and throughput.
%cause performance problems inside the network.
%
Even worse, the open and customizable nature of mobile OSes, such as
Google Android~\cite{Android}, can be further abused by misbehaving
devices/users to gain an unfair advantage in throughput performance
by manipulating the tethering function.
%
For example, by rooting or jailbreaking mobile OSes, the channel access
functions of Wi-Fi protocol, i.e., IEEE 802.11, can be manipulated
``selfishly,'' at the cost of other nearby well-behaving Wi-Fi
devices' performance.
%
Therefore, it is essential to design an efficient mechanism to detect
unauthorized and misconfigured Wi-Fi tethering.

In this paper, we observe the network throughput behavior in a multi-AP
environment in the presence of a selfish tethering node, and study the
impact of selfish tethering on other users' throughput performance.
%
Our evaluation results reveal that the symptoms of selfish tethering
resembles that of the well-known hidden node problem---both result in a
high frame transmission error rate.
%
However, the main difference is that the frame losses at legitimate nodes
caused by the selfish tethering cannot be easily resolved by the RTS/CTS
mechanism because the selfish nodes may not recognize the RTS/CTS frames
due to their manipulated CCA thresholds, or RTS/CTS frames can be easily
ignored by the selfish nodes.
%
Therefore, it is imperative to design a detection mechanism that can
identify the root cause of the throughput degradation and pinpoint the
selfish node in the network.

Based on our observations, we present an online Wi-Fi tethering
detection mechanism, called {\tt CUBIA}, that can accurately detect
selfish Wi-Fi tethering nodes in a multi-AP environment.
%
{\tt CUBIA} runs at each AP and passively monitors the channel-access
success/failure behavior of ongoing data traffic to detect any abnormal
changes caused by selfish tethering.
%
In particular, it monitors any noticeable increase in frame loss rate to
trigger the second-phase detection mechanism, where it attempts to identify
the root cause of the frame loss, e.g., selfish tethering or
hidden node problem.
%
For this, we formulate the selfish tethering detection problem as a
hypothesis testing problem and employs a modified CUSUM algorithm.
%
% alex: need to revise below based on evaluation results
We evaluate {\tt CUBIA} with an in-depth simulation in various
wireless environments, and our evaluation results show high
detection accuracy.
%
To best of our knowledge, this is the first to consider the problem
of detecting selfish misconfigured Wi-Fi tethering nodes in
a multi-AP network.


\subsection{Contributions}

This paper makes several contributions as follows.
%
\myitemizebegin
%
\item We observe that concurrent transmission mechanisms, e.g.,
PHY capture or MIM, can be abused by selfish nodes, and study
the impact of the selfish Wi-Fi tethering behavior on
the throughput performance of other nearby well-behaving nodes.
%
\item We propose a selfish misbehavior detection mechanism,
called {\tt CUBIA}, that can accurately identify the cause of
frame losses and detect selfish behavior under strict detection
latency requirements.
%
\item We design a two-step detection process using an online
change detection algorithm, i.e., CUSUM, based on passive
monitoring of frame loss rates at multiple APs. We also
study the impact of detection thresholds on detection
latency/accuracy performance.
%
\item We perform extensive simulation-based evaluation for
various network conditions. Our evaluation results show that
{\tt CUBIA} can promptly detect selfish behavior with high
accuracy in realistic wireless environments.
%
\myitemizeend

\subsection{Related Work}

Selfish and malicious misbehavior in CSMA networks has been
extensively studied under various scenarios in different 
communication layers.
%~\cite{K:Vaidya05, Radosavac-Wise05, Raya-DOMINO, Toledo:Wang07a}.
%
%MAC-layer
Majority of previous work has concentrated on detecting or punishing
MAC-layer misbehavior~\cite{K:Vaidya05, Raya-DOMINO, Radosavac-Wise05,
Toledo:Wang07a}.
%
In~\cite{K:Vaidya05}, Kyasanur and Vaidya studied the MAC-layer
misbehavior of selecting always small backoff values rather than
random selection, and showed that such selfish misbehavior can
seriously degrade the network performance.
%
Raya {\em et al.}~\cite{Raya-DOMINO} investigated multiple misbehavior
policies in 802.11 MAC protocol including backoff manipulations, and
presented a detection framework, called {\tt DOMINO}, which considers
all possible strategies jointly and improves the detection accuracy.
%
Several statistic-based frameworks for detecting misbehavior
also have been introduced~\cite{Radosavac-Wise05, Toledo:Wang07a}.
%
%The common approach~\cite{K:Vaidya05, Radosavac-Wise05, Raya-DOMINO,
%Toledo:Wang07a} is to measure the inter-arrival time of target
Their common detection approach is to measure the inter-arrival time
of target stations in terms of the number of backoff slots and verify
whether the backoff time of the stations follows a legitimate pattern
or not.
%
Recently, the authors in~\cite{Tang:Zhuang13} proposed a
non-parametric CUSUM test to detect real-time selfish misbehavior 
in 802.11 networks.
% alex: what is the main difference between our work and Zhuang13?
%
The problem of selfish misbehavior also has been addressed in several
different layers and perspectives, including routing
layers~\cite{BenSalem03, Marti03}, multi-channel
protocols~\cite{Zhang:Lazos13}, and game theory-based
behaviors~\cite{MacKenzie01,Akella02}.

%%
The problem of misbehavior using CCA threshold is relatively new 
and unexplored in the literature. In~ \cite{yang:choi-LCN09}, the
authors have identified the effect of an 802.11 node's selfish
behavior by increasing CCA threshold.
%
They have shown that the selfish behavior can achieve higher
throughput than other well-behaving nodes based on a game-theoretical
model.
%
Recently, the authors in~\cite{Pelechris:Krish-Infocom09} addressed
the selfish carrier sense problem in a Wireless LAN (WLAN).
The authors have performed experimentation on a real-world testbed 
and shown that such selfish behaviors can cause extremely unfair
allocations of the wireless medium.
%
However, none of these studies considered the problem of the selfish
problem exploiting Wi-Fi tethering environments.


\subsection{Organization}
%
The remainder of paper is organized as follows.
%
Section~2 describes the system model and adversary model.
%
Section~3 illustrates the impact of selfish manipulation of the
CCA thresholds on throughput performance in multi-AP environments.
%
Section~4 identifies the unique features of selfish tethering and
proposes {\tt CUBIA} that accurately detects the selfish behavior.
%
Section~5 presents the evaluation results and Section~6 concludes 
the paper.

