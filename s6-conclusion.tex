%!TEX root = info-main.tex
\section{Conclusion and Future Work}
%
In this paper, we present {\tt CUBIA}, a novel Wi-Fi tethering misbehavior
detection mechanism that can accurately detect selfish behavior, e.g.,
manipulating CCA threshold, via AP-level collaboration in multi-AP
network environments.
%
We show that the benefits of MIM can be fully exploited and abused further
by a selfish tethering node via CCA manipulation combined with its short
link-distance.
%
We also observe that the consequence of the selfish tethering behavior resembles
that of the hidden node problem, but selfish tethering nodes tend to ignore
the RTS/CTS mechanism.
%
{\tt CUBIA} employs CUSUM algorithm for online detection of abnormal network
behavior and inject RTS/CTS frames to avoid mis-diagnosing the hidden node
problem as a selfish tethering.
%
Our simulation-based evaluation results show that {\tt CUBIA} accurately
distinguishes the selfish tethering behavior from other types of misbehavior
including the hidden node problem. 

In future, we would like to extend our work to pinpoint and localize
the selfish node based on the cooperation among APs.
It would also be interesting to study effective system-level follow-up 
actions against selfish tethering misbehavior, such as 
jamming-resilient dynamic channel re-assignment.

%To the best of our knowledge, this is the first attempt addressing the selfish
%node problem in a tethering environment. 
