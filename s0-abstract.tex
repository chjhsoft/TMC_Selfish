Wi-Fi tethering, using a mobile device (e.g., a smartphone or a
tablet) as a hotspot for other devices, has become a common practice.
%
%However, the tethered Wi-Fi hotspot can potentially cause serious
%performance problems within a well-planned Wi-Fi network due
%to an unauthorized selfish device interfering with nearby
%well-planned APs.
%
%Besides, the open source nature of mobile operating systems
%can be abused to manipulate the parameters that determine
%the channel access probability of a tethering node.
%
Despite the potential benefits of Wi-Fi tethering,
the open source nature of mobile operating systems (e.g., Google
Android) can be abused by a selfish device to manipulate channel-access
parameters to gain an unfair advantage in throughput performance.
%
This can cause serious performance problems within a well-planned
Wi-Fi network due to an unauthorized selfish tethering device
interfering with nearby well-planned APs.
%
In this paper, we show that the selfish behavior of a tethering node
that adjusts the clear channel assessment (CCA) threshold has strong
adverse effects in a multi-AP network, while providing the selfish
node a high throughput gain.
%
To mitigate/eliminate this problem, we present a passive online
detection architecture, called {\tt CUBIA}, that diagnoses the network
condition and detects selfish tethering nodes with high accuracy by
exploiting the packet loss information of on-going transmissions.
%
%CCCC Mention some numbers
To best of our knowledge, this is the first to consider the
problem of detecting a selfish tethering node in multi-AP networks.
